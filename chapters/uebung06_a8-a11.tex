\documentclass{standalone}
\begin{document}

\subsection{Aufgabe 6.8}
\begin{enumerate}
    \item $\begin{pmatrix}
        A = 0
    \end{pmatrix}$
    \item Einfach die Matrix aus Aufgabe 6.11 nehmen und einen beliebigen Winkel einsetzen. Allgemein gilt: $AA^{-1} = I_n$, mit $A \in \mathbb{R}^{3 \bigtimes 3}$. Für die Matrix in Aufgabe 6.11 gilt speziell: $A^{-1}=A^T$
\end{enumerate}
    
\subsection{Aufgabe 6.9}
\begin{enumerate}[a)]
    \item $A = \begin{pmatrix}
        0 & 0 \\
        0 & 1
    \end{pmatrix}\text{ ; }
    B = \begin{pmatrix}
        1 & 1 \\
        0 & 0
    \end{pmatrix}
    $
    \item \begin{align}
        A &= \begin{pmatrix}
            0 & 2 & 4 \\
            0 & 0 & 1 \\
            0 & 0 & 0
        \end{pmatrix}\\
        A^2 &= \begin{pmatrix}
            0 & 0 & 2 \\
            0 & 0 & 0 \\
            0 & 0 & 0
        \end{pmatrix} \\
        A^3 &= 0
    \end{align}
    \item \begin{align}
        B &= \begin{pmatrix}
            0 & 0 & 0 \\
            \frac{1}{2} & 0 & 0 \\
            3 & 0 & 0
        \end{pmatrix}\\
        B^2 &= 0
    \end{align}
\end{enumerate}

\subsection{Aufgabe 6.10}
\begin{enumerate}[a)]
    \item Beispiel: $\vec{v} = (v_1, v_2)^T$. Dann ist $\begin{pmatrix}
        3 & 0 \\
        0 & 3
    \end{pmatrix}
    = (3v_1, 3_v2)$\\
    Die Einträge des Vektors werden verdreifacht, der Vektor damit in derselben Richtung auf sein Dreifaches verlängert.
    \item Die Koordinaten seiner Punkte halbieren sich, somit auch seine Kantenlängen.
    \item Spiegelung an der x-Achse bedeutet $x_2 = x_1$ und $y_2 = -y_1$. Damit ergibt sich $C = \begin{pmatrix}
        1 & 0 \\
        0 & -1
    \end{pmatrix}$
\end{enumerate}

\subsection{Aufgabe 6.11}

\begin{enumerate}[a)]
    \item \begin{align}
        A(\phi)A(\phi)^T &= \begin{pmatrix}
            \cos \phi & -\sin \phi \\
            \sin \phi & \cos \phi
        \end{pmatrix}
        \begin{pmatrix}
            \cos \phi & \sin \phi \\
            -\sin \phi & \cos \phi
        \end{pmatrix}
        \\ &= 
        \begin{pmatrix}
            \cos^2 \phi + \sin^2 \phi & \cos \phi \sin \phi - \sin \phi \cos \phi \\
            \sin \phi \cos \phi - \cos \phi \sin \phi & \sin^2 \phi + \cos^2 \phi
        \end{pmatrix}
        \\ &=
        \begin{pmatrix}
            1 & 0 \\
            0 & 1
        \end{pmatrix}
        \\ &= I_2
    \end{align}
    \item Sie dreht den Punkt um $\phi$ um den Koordinatenursprung, gegen den Uhrzeigersinn.
    \item \begin{align}
        A(\phi)A(\theta) &= \begin{pmatrix}
            \cos \phi \cos \theta -\sin \phi \sin \theta & -\cos \phi \sin \theta - \sin \phi \cos \theta \\
            \sin \phi \cos \theta +\sin \phi \cos \theta & -\sin \phi \sin \theta + \cos \phi \cos \theta
        \end{pmatrix}
        \\ &= \begin{pmatrix}
            \cos (\phi + \theta) & -\sin (\phi + \theta) \\
            \sin (\phi + \theta) & \cos (\phi + \theta)
        \end{pmatrix}
        \\ &= A(\phi + \theta)
    \end{align}
\end{enumerate}

\end{document}