\documentclass{standalone}
\begin{document} 

\subsection{Aufgabe 4.3}
% Hier eure Lösung eingeben

\subsection{Aufgabe 4.4}
% Hier eure Lösung eingeben

\end{document}