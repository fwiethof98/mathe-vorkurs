\documentclass{standalone}
\begin{document}

\subsection{Aufgabe 6.6}
\begin{enumerate}[a)]
	\item
		Ist $M$ eine $(n \times m)$-Matrix, dann hat die Transformation $M^T$ die Form 
		$(m \times n)$. Falls $M = M^T$ gilt, dann geht dies nur, wenn $m = n$ gilt. $\Rightarrow (n \times n)^T = (n \times n)$. 
	\item
		\begin{itemize}
			\item[A:] Ist symmetrisch, da sich 1 und 1 gegenüber liegen $\Rightarrow A = A^T$ \checkmark
			\item[B:] Ist ebenfalls symmetrisch, da sich -1 und -1 gegenüber liegen. \checkmark
			\item[C:] Ist nicht symmetrisch, da $C_{3, 2} = -4 \neq C_{2,3} = 4$. \ding{55} 
		\end{itemize}
\end{enumerate}
\end{document}